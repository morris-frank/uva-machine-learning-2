\documentclass{article}
\usepackage{morris}
\usepackage{diagbox}
\usepackage{centernot}
\usepackage{tikz}
\usepackage{enumerate}

\newcommand{\indep}{\mathrel{\text{\scalebox{1.07}{\ensuremath{\perp\mkern-10mu\perp}}}}}
\newcommand{\nindep}{\centernot{\indep}}

\renewcommand{\thesection}{Problem \arabic{section}.}
\renewcommand{\thesubsection}{\arabic{subsection}.}
\renewcommand{\thesubsubsection}{\alph{subsubsection})}

\title{Machine Learning 2 --- Homework 4}
\author{%
  Maurice Frank\\
  11650656\\
  \href{mailto:maurice.frank@posteo.de}{maurice.frank@posteo.de}
}

\begin{document}
\maketitle
\section{}
We have \(\B{X} = \{x_1, \…, x_N\}\) and \(\B{Z} = \{z_1, \…, z_N\}\).

\subsection{}
\begin{align*}
  p(\B{Z}, \B{X})
  &= p(z_1)\·\left(\Π_{i=2}^N p(z_i|z_{i-1})\right) \· \left(\Π_{i=1}^N p(x_i|z_i)\right)
\end{align*}

\subsection{}
We present the factor graph for the Markov chain:
\begin{center}
  \tikzstyle{var} = [circle, draw=red, inner sep=2pt, thick]
  \tikzstyle{func} = [rectangle, fill, red]
  \begin{tikzpicture}
    \path (0,0) node[var](x1) {\(x_1\)} (2,0) node[var](x2) {\(x_2\)} (6,0) node[var](xN) {\(x_N\)};
    \path (0,0.75) node[func](f_x1) {} (2,0.75) node[func](f_x2) {} (6,0.75) node[func](f_xN) {};
    \path (0,1.5) node[var](z1) {\(z_1\)} (2,1.5) node[var](z2) {\(z_2\)} (6,1.5) node[var](zN) {\(z_N\)};
    \path (0,2.25) node[func](f_z1) {} (1,2.25) node[func](f_z2) {} (3,2.25) node[func](f_z3) {} (5,2.25) node[func](f_zN) {};
    \path (3.7,1.7) node(z3) {} (4.3,1.7) node(zN1) {};
    \draw[red, thick] (x1) -- (f_x1) -- (z1) (x2) -- (f_x2) -- (z2) (xN) -- (f_xN) -- (zN) (z1) -- (f_z2) -- (z2) -- (f_z3) (z1) -- (f_z1) (zN) -- (f_zN);
    \draw[red, thick,dotted] (f_z3) -- (z3) (zN1) -- (f_zN);
    \path (4,1) node[red]() {\(\…\)};
  \end{tikzpicture}
\end{center}

\subsection{}
\begin{align*}
  p(\B{X}) &=
  f_1(z_1) \· \left(\Π_{i=2}^N f_i(z_i, z_{i-1})\right) \· \left(\Π_{i=1}^N f_{N+1}(z_i, x_i)\right)
\end{align*}

\subsection{}
\begin{align*}
  p(z_n|\B{X})
  &= \÷{\α(z_n) \β(z_n)}{p(\B{X})}
\end{align*}

\section{}
\subsection{}
\begin{align*}
  \μ_\α(x_2) &= \Σ_{x_1} \ψ_{1,2}(x_1,x_2)
  \μ_\α(x_i) &= \Σ_{x_{i-1}} \ψ_{i-1,i}
\end{align*}

\subsection{}

\section{}

\section{}

\end{document}
